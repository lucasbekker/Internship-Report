\chapter{Central processing unit}

\section{Basics of a CPU}

\hspace{4mm}Processors, or CPU's, are the "beating heart" of a computer, and as such, perform many more tasks than the ones that will be discussed in this text. The focus lies on the floating point calculation capabilities and the memory subsystems, as well as the interfaces.\vspace{5mm}

The traditional CPU\autocite[]{wiki_cpu} is an integrated circuit that executes the logic, arithmetic , input/output (I/O) and control operations that are prescribed by software running on the computer. As time passed, many other subsystems of computers got integrated into the processor package (die), making the functions that the traditional CPU performs only a subset of all the functions that a modern processor performs. A "core" of a modern processor is a separate unit that performs all the tasks of a traditional CPU.\vspace{5mm}

Modern processor cores are very diverse, complex and multi-faceted, varying wildly with ISA, microarchitecture, intended platform and manufacturer. A discussion about CPU's that would include all these variations would be impossible, necessitating a confinement. This discussion will try to be as generic as possible, but the focus lies on an Intel based server CPU of the "Skylake-SP" microarchitecture.\vspace{5mm}

Intel dominates the PC/laptop as well as the HPC/Supercomputer CPU market, making the restriction towards an Intel x86-64 based CPU justified. Considering the fact that almost all Laptops and workstations contain CPU's that are (to a varying extend) derived from their server oriented counterparts, focusing the discussion around a server CPU seems logical as well. The Skylake-SP microarchitecture was chosen because it is very recent (at the time of writing), contains some very significant advancements for scientific computing workloads and is used in a cluster available to the MEFD group at the TU/e.

\newpage

\section{System on a Chip}

\hspace{4mm}Modern processors are best described by the "System on a Chip"\autocite[]{wiki_cpu} moniker, containing many of the core components of a computer. As such, most contain the following subsystems:

\begin{itemize}
    \item Cores
    \item Memory controller
    \item Cache
    \item Interfaces
    \item Graphics (included in most consumer oriented CPU's)
\end{itemize}

\section{Instruction set architecture}

\hspace{4mm}An Instruction Set Architecture is an abstract model of a computer and contains a collection of machine instruction definitions\autocite[]{wiki_isa}. Examples of common ISA's are x86-64, x86 and ARM, with x86-64 being the most common ISA for CPU's in servers, as well as consumer oriented computers.\vspace{5mm}

An ISA is one of the most important aspects of a CPU, because it forms the link between software and hardware. ISA's where introduced to make programming software easier, which could now be written in terms of ISA instructions in stead of low level machine code\autocite[]{wiki_mc}. This made it possible to execute the same computer program on different computers, without any modification of the code.\vspace{5mm}

An implementation of an ISA, called a microarchitecture (uarch)\autocite[]{wiki_ua}, is the hardware based realization of these machine instruction definitions (disregarding microcode\autocite[]{wiki_uc}). Any specific uarch can also support extensions to its ISA, common examples are VT-d, AES-NI, SSE4 and AVX2. These extensions are additions to the abstract computer model of an ISA and contain specific instructions to accelerate certain tasks of a computer, like AES data encryption, virtualization and vector mathematics.\vspace{5mm}

Some better known examples of microarchitectures are Intel i386, Intel Nahelem, Intel Haswell, AMD K8, AMD Bulldozer and AMD Zen. 

\newpage

\section{Threads and cores}

\hspace{4mm}A thread is a chain of instructions that is to be executed by the CPU\autocite[]{wiki_thread}. Each thread is generated by a process, which can loosely be described as an instance of a computer program. A single core of a CPU can execute one thread at a time, but using a technique called time slicing and the concept of context switching\autocite[]{wiki_cs}, can handle multiple threads concurrently.\vspace{5mm}

Multithreading (software) allows a single process to spawn a multitude of threads, dividing the workload of that process. Performance benefits (can) arise when these threads are executed in parallel on multiple cores of a CPU.\vspace{5mm}

The specifications of a CPU may contain references to the number of threads it "has", which should be interpreted as the maximum amount of threads it can "execute" at the same time. The fact that a single CPU core can only execute one thread at a time doesn't change, even if the CPU specifications state that it has more (twice) threads then cores. This has to do with a hardware based technique called simultaneous multithreading\autocite[]{wiki_smt}, which will be discussed later.

\section{Cache}

\hspace{4mm}Quick access to data is critical for the performance of a CPU, making data flow and storage a mayor aspect of a CPU. The main memory of a computer has a relatively high latency and low bandwidth compared to the needs of modern CPU cores, which is where cache comes into play\autocite[]{wiki_cache}.\vspace{5mm}

Cache is storage subsystem of the CPU and acts like a data buffer. It contains, among other things, copies of the data that the CPU (or process) "predicts" it will access often or in the near future, reducing the loading time of that data. The amount of cache placed on the CPU is relatively small, typically about 1:1.000, compared to the amount of main memory placed in a computer. A "cache-miss" refers to the situation where data is requested, but not stored in cache, resulting in a much longer loading time. Avoiding cache-misses is a large part of software (and hardware) optimization and can lead to very substantial performance improvements.\vspace{5mm}

Cache memory pressure (memory nearly full) and the ever increasing speed of CPU cores lead to the development of multiple levels of cache. This layered structure has the advantage that it can address both the memory pressure problem as well as the demand for faster data access, without resulting in prohibitive costs. The upper most layer of cache, L1, has gotten significantly faster over time, but did not really increase much in capacity. The lowest level of cache, typically L3, saw the highest increase in capacity, but is also substantially slower then L1.\vspace{5mm}

The development of multi-core processors and several levels of cache created an additional task for cache; inter-core data communication. Each core has its own private parts of L1 and L2 cache, whereas L3 cache is shared between the cores. This last level of cache is the place where data can be shared between the threads running on the different cores.\vspace{5mm}

\newpage

Cache memory is a lot faster, and generally superior on many fronts, compared to main memory, because cache is made from SRAM and main memory is made from DRAM. SRAM stands for "Static Random Access Memory", whereas DRAM stands for "Dynamic Random Access Memory". Each SRAM memory cell requires 6 transistors to store a bit, whereas DRAM requires only one transistor (and a small capacitor) per bit. The downside of DRAM is that the capacitors in DRAM memory need to be recharged frequently, causing delays and other problems. This constant refreshing of the stored data gave rise to the name "Dynamic", while "Static" was used for SRAM, because it doesn't need to be refreshed. The extra hardware complexity of SRAM allows it to be much faster than DRAM, but the extra cost and space requirements on the die of the CPU also make it much more expensive.

\section{Execution units}

\hspace{4mm}Execution units of a CPU core are the parts that execute the machine instructions derived from the thread running on the core. There are many different types of execution units in modern CPU cores, each with their own specific function. Notable examples of execution units are; arithmetic logic unit\autocite[]{wiki_alu}, address generation unit\autocite[]{wiki_agu} and floating-point unit\autocite[]{wiki_fpu}. Discussing the functions and operations of all these execution units is beyond the scope of this text, which will focus on the floating-point execution unit.

\subsection*{Superscalar}

\hspace{4mm}CPU cores have many execution units, most also have multiple execution units of the same type. Keeping all the execution units busy at the same time requires multiple instructions to be dispatched (one instruction per execution unit) simultaneously. The ability of a CPU core to dispatch multiple instructions simultaneously is called being superscalar\autocite[]{wiki_super}, whereas CPU's that can only dispatch a single instruction are called scalar. Superscalar capabilities are a form of instruction-level parallelism\autocite[]{wiki_ilp}.

\subsection*{SIMD}

\hspace{4mm}SIMD\autocite[]{wiki_simd} stands for single instruction, multiple data and is a form of data level parallelism\autocite[]{wiki_dlp}. SIMD is a vector processing technique, allowing an execution unit to perform the same instruction on multiple data entries (grouped in 1D arrays called vectors) in a single clock cycle. The maximum achievable throughput increases substantially by using SIMD, but does requires all the data to be manipulated in the same way, making it less versatile.\vspace{5mm}

SIMD works by exposing deep registers to execution units, containing the data vectors\autocite[]{wiki_register}. The instruction that the execution unit receives is performed on the complete register.

\newpage

\subsection*{FMA3 and advanced vector extensions}

\hspace{4mm}The floating-point execution units found in modern Intel CPU cores are based on FMA3 (from Haswell on wards). These FMA based units are capable of three different operations:

\begin{itemize}
    \item $ a = a \cdot c + b $
    \item $ a = b \cdot a + c $
    \item $ a = b \cdot c + a $
\end{itemize}

The Skylake-SP uarch contains two AVX-512 execution units, with 512 bit deep registers. Each AVX-512 unit contains 8 FMA3 sub-units for "double" floating-point numbers and 16 FMA3 sub-units for "single" floating-point numbers. These AVX-512 execution units form the hardware layer of the AVX-512 ISA extension\autocite[]{wiki_avx512}.\vspace{5mm}

The Haswell and Broadwell uarch contain two AVX2 execution units, with 256 bit deep registers. Each AVX2 unit contains 4 FMA3 sub-units for "double" floating-point numbers and 8 FMA3 sub-units for "single" floating-point numbers. These AVX2 execution units form the hardware layer of the AVX2 ISA extension\autocite[]{wiki_avx2}.\vspace{5mm}

AMD's latest (at the time of writing) Zen architecture also supports AVX2 instructions (not AVX-512), but the hardware based implementation is completely different. It doesn't have native support for 256 bit deep registers and each AVX2 instruction takes 2 clock cycles to complete, compared to one clock cycle of Intel based AVX2 capable CPU's.

\section{Simultaneous multithreading}

\hspace{4mm}Simultaneous multithreading, also known as Hyper-Threading\autocite[]{wiki_ht}, is a technique aimed at increasing the utilization of a CPU core. Each physical CPU core represents multiple logical CPU cores, fully transparent to the Operating System. Every logical CPU core gets their own thread assignment by the OS, meaning that a single CPU core is tasked with multiple threads. The amount of logical cores per physical core differs from microarchitecture to microarchitecture. The most common is two logical cores per physical core, but Intel Xeon Phi\autocite[]{wiki_phi} and IBM POWER9\autocite[]{wiki_p9} based designs have 4 and up to 8 logical cores per physical core.\vspace{5mm}

When a thread (assigned to a CPU core) is unable to assign tasks to every execution unit of the CPU core, instructions of a different thread assigned to the same CPU core can be dispatched to the unused execution units. One of the most common culprits for a thread to under utilize the execution units, is cache misses. SMT can be very helpful in hiding the latency caused by data requests that require multiple clock cycles to fulfill.\vspace{5mm}

The performance improvements generated by SMT vary wildly with applications, from a factor of two down too a performance decrease. SMT has the largest positive effect in situations where cache-misses are frequent, instruction level parallelism per thread is low and the workloads of the threads are very heterogeneous. Math routines provided by highly optimized libraries, such as the Intel Math Kernel Library\autocite[]{wiki_mkl}, generally don't fall into this category, making SMT less beneficial for HPC purposes.\vspace{5mm}

\newpage

\section{Operating frequency}

\hspace{4mm}The (high level) building blocks of digital circuits are called logic gates, which are hardware implementations of boolean operations. These logic gates are combined in intricate ways to provide more high level functionality, like the FMA operation on floating-point data. Synchronization of the logic gates is key for their operation, which is where the clock signal comes into play\autocite[]{wiki_clock}.\vspace{5mm}

The clock signal of the CPU core is a square wave signal, switching between high and low (logical "on" and "off"). The rising and falling edges of this square wave are the "timing" signals for the logic gates to evaluate their input. The frequency of this timing signal is called the operating frequency or clock frequency of the CPU core. The operating frequency of a CPU core is directly linked to the throughput of micro-operations\autocite[]{wiki_uop}, making it a key factor in the performance of a CPU.

\subsection*{Turbo frequencies}

\hspace{4mm}Almost all modern CPU's employ some sort of dynamic operating frequency control, allowing the operating frequency of various parts of the CPU to go up or down in conjunction with demand and thermal headroom. Dynamic scaling of the operating frequencies took a flight when mobile devices became more popular, requiring momentary high performance and long battery life. The basic idea behind these techniques is that a relatively high operating frequency can be achieved for a short duration of time. This increases performance for workloads that can be completed within the time frame of the elevated operating frequency, but doesn't significantly increase the overall heat production and power consumption of the CPU. The turbo boost technology of modern CPU's is too complex to discuss in great detail in this text, but a few important aspects pertaining to floating point performance will be explained.\vspace{5mm}

Specifications of the turbo frequencies are very important to the performance of a CPU, but are often reduced to a single number, masking the complete story for marketing purposes. Turbo frequencies scale down according to the workload type (normal, AVX2 and AVX512) and the amount of active cores. AVX512 workloads produce the most heat and highest power consumption because their execution units contain the most transistors, AVX2 execution units require less power and normal (non floating-point) workloads even less than that.\vspace{5mm} 

A more detailed specification of the turbo frequencies of an Intel Xeon Gold 6132 CPU will be provided as an example. Intel ark based information on the Xeon Gold 6132 specifies a base frequency of 2.6 GHz and a maximum turbo boost of 3.7 GHz\autocite[]{ark_xg6132}. WikiChip provides more details on the frequencies section of the Xeon Gold 6132\autocite[]{wchip_6132}. The highest floating-point performance of the Xeon Gold 6132 is achieved when all cores are utilizing their AVX-512 execution units. The maximum turbo frequency of this workload is 2.3 GHz, which is about 40\% lower than the maximum frequency (3.7 GHz) provided by Intel Ark.

